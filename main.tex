\documentclass{beamer}
\mode <presentation>
{
    \usetheme{boxes}
    \usecolortheme{crane}
    \setbeamercovered{transparent}
}

\usepackage[absolute,overlay]{textpos}
\usepackage{pgf,pgfarrows,pgfnodes}
\usepackage[english]{babel}
\usepackage{lmodern}
\usepackage{newcent}
\usepackage{amsmath}
\usepackage{listings}
% math extension - one probably wants to use symbols like '[' (written as '$[$')
\usepackage{ucs}
%\usepackage[utf8]{inputenc}
%\usefonttheme{structuresmallcapsserif}

% utf8x does not work with xetex
\usepackage[utf8x]{inputenc}

\usepackage[normalem]{ulem}


\setlength{\TPHorizModule}{1mm}
\setlength{\TPVertModule}{1mm}
\newcommand{\WorkInProgress}{%
\begin{textblock}{14}(120.0,75.7)
\includegraphics[height=0.1cm]{./pics/rmll2011.jpg}
\end{textblock}
  }

%\setbeamercolor{background canvas}{bg=\includegraphics[width=\textwidth]{./pics/wolf.png}}

\title{The FusionInventory project}
\author{{FusionInventory.org}}
\subject{Assets management with FusionInventory and GLPI}
\keywords{Assets management, Inventory, FusionInventory, GLPI}

\date{Juillet 2012}
%\titlegraphic{GLPI}
%subtitle{\includegraphics[width=1.2cm]{./pics/fusioninventory-logo.png}}
\institute{\includegraphics[height=4.3cm]{./pics/rmll2011.jpg}}

\titlegraphic{}
\subtitle{RMLL / LSM 2012}
\institute{Genève}
\author{ Walid Nouh - Mathieu Simon}
\logo{\includegraphics[height=0.7cm]{./pics/fusioninventory-logo.pdf}}

\AtBeginSection[] % Do nothing for \section*
{
    \begin{frame}<beamer>
        \frametitle{Outline}
        \tableofcontents[currentsection]
    \end{frame}
}

%%%%%%%%%%%%%%%%%%%%%%%%%%%%%%%%%%%%%%%%%%%%%%%
%%%%%%%%%%%%%%%%%%%%%%%%%%%%%%%%%%%%%%%%%%%%%%%
\begin{document}

\frame[plain]{\titlepage}


\begin{frame}
    \frametitle{About us}


    \begin{block}{Walid Nouh}
        \begin{itemize}
        \item FusionInventory contributor
        \item GLPI contributor  
        \item Working for TECLIB', Montpellier
        \end{itemize}
    \end{block}

\end{frame}

\begin{frame}
    \frametitle{About us}


    \begin{block}{Mathieu Simon}
        \begin{itemize}
        \item FusionInventory contributor (l10n)
        \item GLPI plugin translation
        \item Working for Gymnasium Köniz-Lerbermatt, Bern
        \end{itemize}
    \end{block}

\end{frame}

\section{Project overview}

\begin{frame}
    \frametitle{FusionInventory timeline}

    \begin{description}
      \item[2006] First unified inventory agent for Unix
      \item[2008] First server implementation (Tracker, a plugin for GLPI)
      \item[2009] Agent and server integration
      \item[2010] FusionInventory project is born! 
      \item[2010] Uranos integration
      \item[2011] Rudder (cfengine) integration
      \item[2012] OTRS integration
    \end{description}

\end{frame}



\begin{frame}
    \frametitle{The project}
    %%-------------------------------------------------------------------
    %%\logo{\includegraphics[height=3.5cm]{./pics/glpi-doc.png}}
    FusionInventory is a community driven project

    \begin{itemize}
        \item Active mailing list
        \item IRC: \#FusionInventory on FreeNode
        \item Forge, Git repositories, etc
    \end{itemize}
\end{frame}


\begin{frame}
    \frametitle{Contributors}

 \begin{columns}
 \begin{column}[T]{4cm}
    \includegraphics[height=3.7cm]{./pics/worker.jpg}
 \end{column}
 \begin{column}[t]{5cm}
    \begin{itemize}
    \item 4 active developers
    \item An active community
    \item 2 companies involved
    \end{itemize}

 \end{column}
\end{columns}

    \pause
    \bf{We're looking for more contributors !}
\end{frame}



\begin{frame}
    \frametitle{Before starting}

    \begin{block}{FusionInventory is not a software}
    \begin{itemize}
        \item Agent: a software to install on the computers
        \item Server: handles communication with the agents
        \item Task: is prepared by the server, executed by an agent
    \end{itemize}
    \end{block}

\end{frame}

\begin{frame}
    \frametitle{pull / push}

    \begin{block}{FusionInventory enables "push" or "pull" mode}
    \begin{itemize}
    \item \textbf{"pull": Agent $\Longrightarrow$ Server} \\
    the agent contacts the server.
    \item \textbf{"push": Agent $\Longleftarrow$ Server} \\
    the server initiates the contact.
    \end{itemize}
    \end{block}

\end{frame}


\begin{frame}
    \frametitle{Available servers today}

    \begin{block}{4 solutions so far}
        \begin{itemize}
            \item FusionInventory for GLPI \\
            \url{http://www.FusionInventory.org}
            \item Uranos \\
            \url{http://uranos.sourceforge.net/}
            \item Rudder by Normation \\
            \url{http://www.normation.com/\#produits}
            \item OCS Inventory NG
            \item Mandriva Pulse 2

        \end{itemize}
        ... it's also possible to perform local XML inventory (soon: switch to JSON).
    \end{block}

\end{frame}

\begin{frame}
    \frametitle{In contact with developers of}

    \begin{itemize}
    \item FusionDirectory
    \item OTRS ITSM (work already in progress)
    \end{itemize}
\end{frame}


\section{GLPI and FusionInventory}

\begin{frame}
    \frametitle{Why GLPI for the server?}

    \begin{itemize}
    \item We're also members of the GLPI community
    \item Motivation: Don't reinvent the wheel!
    \item Avoid having a second database and synchronization
    \item Focus on what's important to us - let GLPI do the rest
    \end{itemize}
\end{frame}

\begin{frame}
    \frametitle{GLPI: about (1/4)}

    \begin{itemize}
    \item Powerfull web based asset management software
    \item Project started in 2003
    \item LAMP
    \item Provides a complete plugin API
    \end{itemize}
\end{frame}

\begin{frame}
    \frametitle{GLPI: inventory (2/4)}

    \begin{itemize}
    \item Computers, screens, printers, etc
    \item Network devices, phones
    \item Financial informations
    \item Warranties
    \item Licenses
    \end{itemize}
\end{frame}

\begin{frame}
    \frametitle{GLPI: Service Desk (3/4)}

    \begin{itemize}
    \item ITIL compatible Servicedesk
    \item Incident management
    \item Problem management
    \item Statistics
    \end{itemize}
\end{frame}

\begin{frame}
    \frametitle{GLPI (4/4)}

    \begin{itemize}
    \item Reports and statistics
    \item Powerful authorization system (LDAP, IMAP/POP, etc)
    \item WebSSO (CAS, LemonLDAP, etc)
    \end{itemize}
\end{frame}

\section{Agent: installation and OS support}

\begin{frame}
    \frametitle{Agent: Installation}


    \begin{block}{Many ways to do it}
        \begin{itemize}
            \item \textbf{Distribution packages} \\
            \small{Debian, Fedora, EPEL, Ubuntu, Mageia, ...}
            \item \textbf{Windows installer} \\
            \small{GPO, psexec, ...}
            \item \textbf{Static prebuilt packages}: untar and run \\
            \small{62 differents system so far}
            \item Tarball and CPAN
        \end{itemize}
    \end{block}
\end{frame}

\begin{frame}
    \frametitle{Agent: Installation}

   \begin{columns}
   \begin{column}{0.35\textwidth}
\includegraphics[height=5.5cm]{pics/googleplay.png}
 \end{column}
 \begin{column}{0.65\textwidth}
There's also an Android agent available on Google Play.
 \end{column}
\end{columns}

\end{frame}


\begin{frame}
    \frametitle{Supported operating systems}

    \begin{itemize}
        \item Linux
        \item Windows
        \item MacOSX
        \item BSD
        \item illumos
        \item AIX
        \item HP-UX
        \item Solaris
        \item Android
    \end{itemize}


\includegraphics[height=0.5cm]{pics/logos/aix.png}
\includegraphics[height=0.5cm]{pics/logos/fedora.png}
\includegraphics[height=0.5cm]{pics/logos/hp-ux.png}
\includegraphics[height=0.5cm]{pics/logos/netbsd.png}
\includegraphics[height=0.5cm]{pics/logos/openbsd.png}
\includegraphics[height=0.5cm]{pics/logos/illumos.png}
\includegraphics[height=0.5cm]{pics/logos/solaris.jpg}
\includegraphics[height=0.5cm]{pics/logos/centos.jpg}
\includegraphics[height=0.5cm]{pics/logos/linux.png}
\includegraphics[height=0.5cm]{pics/logos/osx.png}
\includegraphics[height=0.5cm]{pics/logos/ubuntu.png}
\includegraphics[height=0.5cm]{pics/logos/debian.png}
\includegraphics[height=0.5cm]{pics/logos/freebsd.png}
\includegraphics[height=0.5cm]{pics/logos/redhat.png}
\includegraphics[height=0.5cm]{pics/logos/windows.jpg}
\includegraphics[height=0.5cm]{pics/logos/dragonflybsd.png}
\includegraphics[height=0.5cm]{pics/logos/mageia.png}

\end{frame}

\section{Task: Network discovery}

\begin{frame}
    \frametitle{Network discovery}

    \begin{block}{Quickly detect and collect all connected devices}
    \begin{itemize}
      \item nmap
      \item NetBIOS
      \item SNMP queries
    \end{itemize}
    \end{block}

\end{frame}

\section{Task: network inventory}
%\begin{frame}
%    \frametitle{Remote SNMP inventory}
%
%    \begin{block}{Network devices}
%        \begin{itemize}
%            \item serial number, firmware, ...
%            \item ports mapping
%        \end{itemize}
%    \end{block}
%
%    \begin{block}{Network printers}
%        \begin{itemize}
%            \item serial number, firmware, ...
%            \item cartridge ink level
%            \item page counter
%        \end{itemize}
%    \end{block}
%\end{frame}

\begin{frame}
    \frametitle{... INTERLUDE ...}

%   \begin{columns}
%   \begin{column}{0.35\textwidth}
         \includegraphics[height=7.5cm]{./pics/montagne.jpg}
% \end{column}
% \begin{column}{0.65\textwidth}
%    \begin{block}{... INTERM\`{E}DE ...}
%    \end{block}

% \end{column}
%\end{columns}
\end{frame}


\begin{frame}
    \frametitle{SNMP}

%    \begin{center}
%    \includegraphics[height=4.0cm]{pics/napoleon.jpg}
%    \end{center}

    \begin{block}{SNMP origin}
    \begin{itemize}
    \item A standard \\
    \small{First RFC in 1988}
    \item Designed to monitor equipments
    \item 3 versions 1, 2c, 3 (Cyphering)
    \item OID: Information location
    \item MIB: A collection of OIDs
    \end{itemize}
    \end{block}
\end{frame}

\begin{frame}
    \frametitle{SNMP: what for in FusionInventory?}

    \begin{block}{How do we use SNMP?}
    \begin{itemize}
    \item Identify remote devices (network equipments, printers, ...)
    \item Perform a remote inventory
    \item Get the most important informations
    \end{itemize}
    \end{block}
\end{frame}

\begin{frame}
    \frametitle{SNMP: a nightmare}

    \begin{block}{“Please support my hardware, here is the MIB!”}

    \pause
    \begin{itemize}
    \item Might be hard to find
    \item Often no free or not redistributable 
    \item Important informations might be missing
    \item But worth, they may be wrong !
    \end{itemize}
    \end{block}

%    \begin{block}{With our SNMP models, we are sure device is well supported!}
%    \end{block}
\end{frame}
\begin{frame}
    \frametitle{SNMP: an example}

 \begin{columns}
 \begin{column}{0.35\textwidth}
         \includegraphics[height=7.5cm]{./pics/cisco.jpg}
 \end{column}
 \begin{column}{0.65\textwidth}
    \begin{block}{Example: Cisco 6500 firmware}
    12.2(33)SXI\textbf{2a} (02-Sep-09 01:00)
    \begin{itemize}
    \item Serial OID: .1.3.6.1.2.1.47.1.1.1.1.11.\textbf{1}
    \end{itemize}
    12.2(33)SXI\textbf{3} (27-Oct-09 11:12)
    \begin{itemize}
    \item Serial OID: .1.3.6.1.2.1.47.1.1.1.1.11.\textbf{2}$\Longleftarrow$ Gni?!
    \end{itemize}
    \end{block}

 \end{column}
\end{columns}


    
\end{frame}

\begin{frame}
    \frametitle{SNMP: outch}

 \begin{columns}
 \begin{column}{0.35\textwidth}
         \includegraphics[height=7.5cm]{./pics/dead-teletubbies.jpg}
 \end{column}
 \begin{column}{0.65\textwidth}

 \end{column}
\end{columns}
\end{frame}



\begin{frame}
    \frametitle{SNMP: how to be reliable ?}

    \begin{block}{We prepared our own “MIB”}
    \begin{itemize}
    \item Manual work for each equipment
    \item stored in an XML file
    \item Defines relations between an OID and an information \\
        \small{ex: serial number → OID 1.2.4.34.53...}
    \item Supports dynamics OIDs
    \end{itemize}
    \end{block}


\end{frame}

\begin{frame}
    \frametitle{... END OF INTERLUDE ...}

    \includegraphics[height=7.5cm]{./pics/montagne.jpg}

\end{frame}



\begin{frame}
    \frametitle{SNMP: network equipments (1/3)}

    \begin{block}{Common informations}
    \begin{itemize}
    \item Serial number
    \item Supplier
    \item Model
    \item Firmware version
    \item MAC address
    \item CPU load / RAM
    \item etc
    \end{itemize}
    \end{block}
\end{frame}

\begin{frame}
    \frametitle{SNMP: network equipments (2/3)}

    \begin{block}{Advanced support}
    \begin{itemize}
    \item Number of ports
    \item Speed
    \item Internal status
    \item Errors counters
    \item VLAN
    \item Trunk (tagged)
    \item ...
    \end{itemize}
    \end{block}
\end{frame}

\begin{frame}
    \frametitle{SNMP: network equipments (3/3)}

    \begin{block}{Port to port connections}
    \begin{itemize}
    \item MAC address \\ 
    \small{one to many}
    \item LLDP / CDP discovery \\
    \small{POIP informations, etc}
    \end{itemize}
    \end{block}
\end{frame}

\begin{frame}
    \frametitle{SNMP: a network equipment example}

    \begin{center}
    \includegraphics[width=11.7cm]{./pics/switch_ports.png}
    \end{center}
\end{frame}

\begin{frame}
    \frametitle{SNMP: Printers (1/2)}

    \begin{block}{General information}
    \begin{itemize}
    \item Serial number
    \item Supplier
    \item Model
    \item Firmware
    \item Memory
    \item MAC Address
    \item etc
    \end{itemize}
    \end{block}
\end{frame}

\begin{frame}
    \frametitle{SNMP: Printers (2/2)}

    \begin{block}{Advanced informations}
    \begin{itemize}
    \item Cartridges state
    \item Pages counters
    \end{itemize}
    \end{block}
\end{frame}

\begin{frame}
    \frametitle{SNMP: a printer example}

    \begin{center}
    \includegraphics[width=11.7cm]{./pics/printer_graph.png}
    \end{center}
\end{frame}

\section{Task: Wake On Lan}

\begin{frame}
    \frametitle{Wake On Lan}

    \begin{block}{WoL}
    \begin{itemize}
        \item Agent can be used as a proxy to send WoL packets.
    \end{itemize}
    \end{block}

\end{frame}

\begin{frame}
    \frametitle{Wake On Lan: Example}

    \begin{block}{Example}
    \begin{itemize}
    \item A remote network
    \item 50 computers
    \end{itemize}
    \end{block}


    \begin{block}{What FusionInventory can do}
    \begin{itemize}
    \item Wake up all computers at 2 am for updates.
    \end{itemize}
    \end{block}

\end{frame}


\section{Task: software deployment}

\begin{frame}
    \frametitle{Software deployment (1/2)}

    \begin{block}{What can be done}
    \begin{itemize}
        \item Perform actions on the target computer
        \item Send files
        \item Consume less bandwith thanks to peer to peer
    \end{itemize}
    Attention: FI is not a configuration management tool !
    \end{block}

\WorkInProgress
\end{frame}

\begin{frame}
    \frametitle{Software deployment (2/2)}

    \begin{block}{Why a new software deployment solution ?}
    \begin{itemize}
        \item Use existing GLPI UI
        \item Use GLPI habilitations (groups/profils/entities)
        \item Multi-platform
    \end{itemize}
    \end{block}

\WorkInProgress
\end{frame}

\section{Task: vCenter / ESX / ESXi inventory}


\begin{frame}
    \frametitle{vCenter / ESX / ESXi}

    \begin{block}{The problem}
    Black boxes: no way to install an agent on the host...
    \end{block}


\end{frame}

\begin{frame}
    \frametitle{vCenter / ESX / ESXi}

    \begin{block}{The solution}
    The agent can use SOAP API to:
        \begin{itemize}
                \item inventorize hardware
                \item list virtual machines per host
                \item inventorize each ESX box (in case of a vCenter)
        \end{itemize}
    \end{block}

\end{frame}

\begin{frame}[fragile]
    \frametitle{vCenter / ESX / ESXi: the command line tool}

\begin{lstlisting}
fusioninventory-esx --host vcenter --user foo \ 
  --password bar --directory /tmp
\end{lstlisting}

Send inventories to the server :
\begin{lstlisting}
fusioninventory-injector -v --file /tmp/*.ocs \ 
  -u https://server/plugins/fusioninventory/
\end{lstlisting}

\end{frame}

\begin{frame}[fragile]
    \frametitle{vCenter / ESX / ESXi: GLPI UI}

 \begin{columns}
 \begin{column}[T]{4cm}
    \includegraphics[height=4.0cm]{pics/esx-glpi.jpg}
 \end{column}
 \begin{column}[t]{6cm}
    \begin{block}{An interface in GLPI}
    \begin{itemize}
         \item Define credentials
         \item Define vCenter/ESX/ESXi address
         \item Plan inventories
    \end{itemize}
    \end{block}
 \end{column}
\end{columns}




\end{frame}

\section{Task: inventory}

\begin{frame}
    \begin{block}{Informations gathered (1/3)}
        \begin{itemize}
        \item BIOS
        \item PCI modules
        \item memory slots
        \item CPUs
        \item hard drivers, drives, etc
        \item motherboard
        \item operating system
        \item screens
        \item ports
        \item slots
        \item partitions
        \item software
        \end{itemize}
    \end{block}
\end{frame}

\begin{frame}
    \begin{block}{Informations gathered (2/3)}
        \begin{itemize}
        \item connected users
        \item video cards
        \item virtual machines
        \item soundcards
        \item modems
        \item environment variables
        \item USB devices
        \item network configurations
        \item batteries
        \item printers
        \item processes
        \item antivirus
        \item LVM
        \end{itemize}
    \end{block}
\end{frame}

\begin{frame}
    \begin{block}{Informations gathered (3/3)}
        Android: Simcard, IMEI , etc
    \end{block}
\end{frame}

\section{QA}

\begin{frame}
    \frametitle{Some statistics}

    \begin{block}{Today}
        \begin{itemize}
            \item 194 Perl modules
            \item 21851 lines of code
            \item 938 unit tests
        \end{itemize}
    \end{block}

\end{frame}


\begin{frame}
    \frametitle{Unit tests}

    \begin{block}{What for?}
        \begin{itemize}
            \item test parsing for OSes that we don't have
            \item check Win32 code from another OS \\
            \small{from WMI to registry}
            \item check sensitive things \\
            \small{unicode, HTTPS, etc}
        \end{itemize}
    \end{block}

\end{frame}

\section{The developer point of view}

\begin{frame}
    \frametitle{What FusionInventory can bring to you}

    \begin{block}{Several scenarios}
        \begin{itemize}
            \item Use inventory in your own application
            \item Extend inventory with your own modules
            \item Interface with GLPI or other \\
            \small{Uranos, soon OTRS, etc}
            \item Create new tasks
        \end{itemize}
    \end{block}

\end{frame}

\section{Soon to come}

\begin{frame}
    \frametitle{What else?}

    \begin{center}
    \includegraphics[height=4.0cm]{pics/whatelse.jpg}
    \end{center}

\end{frame}

\begin{frame}
    \frametitle{Our roadmap}

    Next steps:
    \begin{itemize}
        \item FusionInventory Agent 2.3.x
        \item Tool to edit SNMP XML models
        \item NUT integration
    \end{itemize}

    Transition in progress:
    \begin{itemize}
        \item OCS/XML → REST/JSON
        \small{planned for agent 3.0.0\\already used by OTRS}
    \end{itemize}

\end{frame}

\begin{frame}
    \frametitle{Our roadmap}

    Next steps:
    \begin{itemize}
        \item FusionInventory Agent 2.3.x
        \item Tool to edit SNMP XML models
        \item NUT integration
    \end{itemize}

    Transition in progress:
    \begin{itemize}
        \item OCS/XML → REST/JSON
        \small{planned for agent 3.0.0\\already used by OTRS}
    \end{itemize}

\end{frame}


\section{Real use experience}

\begin{frame}
    \frametitle{Real use experience}

    Why use FusionInventory and GLPI (1/2):
    \begin{itemize}
	\item 2 schools were merged and had grown: 400 each →  1000 students today
	\item IT departments merged and quickly grown
	\item 2008: Regular financial audit - including IT assets.
    \end{itemize}

    We had to get a an inventory without tons of man-hours: QUICKLY.
 
\end{frame}

\begin{frame}
    \frametitle{Real use experience}
    Why use FusionInventory and GLPI (2/2):
    \begin{itemize}
	\item Proprietary solution highly cost intensive - while restricted in features
	\item Experience with superiors: If it's OSS, it has to work
	\item OCS was able to get all our computers' data
	\item GLPI had the nice UI + held all other asset data
    \end{itemize}
\end{frame}

\begin{frame}
    \frametitle{Real use experience}
    OCSinventory-ng side:
    \begin{itemize}
	\item OCS lagged behind with Windows compatibility, regular GLPI sync problems
    \end{itemize}

    Why FusionInventory:
    \begin{itemize}
	\item Client: Agent easier to deploy 3 platforms
	\item Server: PHP + Perl →  PHP
	\item Deployment: 1 internal git tree of GLPI + plugins
	\item IT staff: 2 UIs →  1 UI
    \end{itemize}
\end{frame}

\begin{frame}
    \frametitle{Real use experience}
    OCS Agent: the dull duck - sorry guys...
    \begin{itemize}
	\item OCS lagged behind with Windows compatibility, regular sync problems
    \end{itemize}

    Why FusionInventory:
    \begin{itemize}
	\item Client: Agent easier to deploy 3 platforms
	\item Server: PHP + Perl →  PHP
	\item Deployment: 1 internal git tree of GLPI + plugins
	\item IT staff: 2 UIs →  1 UI
    \end{itemize}
\end{frame}

\begin{frame}
    \frametitle{The good, the bad ...}
    There are also downsides - nothing is perfect:
    \begin{itemize}
	\item Documentation mostly fr-FR focused
	\item l10n: de-DE had quite some "frenchisms"
	\item FI for GLPI lags behind major GLPI releases
	\item Ticket notifications require lots of local tuning (yet)
    \end{itemize}
    ... no: There is no true ugly thing I found
\end{frame}

\begin{frame}
    \frametitle{Finally}
    \begin{itemize}
	\item GLPI + FI automated lots of taks no sysadmin wants to do manually with +300 computers
	\item We still use a fraction of GLPI's features, but growing (Cert management!)
	\item Auditability →  Made superiors happy
	\item OSS proved to work: Swiss-german education - normally not that easy
	\item Got in touch with nice community
	\item Have to polish my french regularly ;-)
    \end{itemize}
\end{frame}

\section{Questions}

\begin{frame}
    \frametitle{Questions?}

% \bf{Questions?}
    \begin{center}

    \includegraphics[height=5cm]{./pics/question.pdf}

    \end{center}
% \includegraphics[height=7.5cm]{./pics/ask.jpg}

\end{frame}

\begin{frame}
    \frametitle{Thanks}

    \begin{block}{Thanks!}
        \begin{itemize}
            \item Windows \url{http://www.flickr.com/photos/aeu04117/430338509/sizes/z/in/photostream/}
            \item AIX \url{http://www.flickr.com/photos/pchow98/5115638572/}
            \item MacOSX \url{http://www.flickr.com/photos/adriannier/5555516312/sizes/l/in/photostream/}
            \item Cisco 6500 \url{http://www.flickr.com/photos/joachim\_s\_mueller/3084164647/sizes/z/in/photostream/}
            \item Teletubbies \url{http://www.flickr.com/photos/tudor/232849285/lightbox/}
            \item Worker \url{http://www.flickr.com/photos/wsdot/6783674428/sizes/l/in/photostream/}
            \item Bee \url{http://www.flickr.com/photos/8583446@N05/7454903214/sizes/l/in/photostream/}
            \item Montagne \url{http://www.flickr.com/photos/blmiers2/6167391543/sizes/l/in/photostream/} 
        \end{itemize}
    \end{block}
\end{frame}




\end{document}
